\chapter{Kết quả thực nghiệm}
\label{chap:result}

\section{Cơ sở so sánh thuật toán tiến hóa}
Các thực nghiệm được thiết kế để chứng minh độ hiệu quả của MFEA-II trong bài toán huấn luyện nhiều mạng ANN khác cấu trúc và mô hình ANN biểu diễn mạng policy của RL. Nhằm đánh giá hiệu quả của giải thuật đề xuất áp dụng MFEA-II, đồ án thực nghiệm so sánh với các giải thuật tiến hóa, tiến hóa đa nhiệm thông thường là CEA và MFEA-I. 
\section{Các bộ thực nghiệm}
\subsection{Mạng neual khác cấu trúc}
\subsubsection{Bộ dữ liệu thực nghiệm}
Bài toán chính để thực nghiệm là bài toán n-bit. Bài toán có đầu vào là một chuỗi bit độ dài n, yêu cầu đầu ra là xác định số bit 1 trong dãy là chẵn hay lẻ. Trong [53], tác giả thực nghiệm với các bài toán 6-bit, 7-bit và 8-bit đối với các mô hình mạng 1 lớp ẩn. Đồ án thực nghiệm với các mạng sâu nên cần tăng độ phức tạp của bài toán hơn nữa. Vì vậy luận văn đề xuất thực hiện giải các bài toán 8-bit, 9-bit và 10-bit. Ngoài ra, thực nghiệm còn đánh giá trên 4 bộ dữ liệu phân loại nhị phân thực tế của \emph{Đại học California, Irvine} (thuật ngữ gốc: University of California, Irvine - UCI) bao gồm: Breast cancer (9 đơn vị đầu vào), Tic-tac-toe (9 đơn vị đầu vào), Ionosphere (34 đơn vị đầu vào) và Credit screening (14 đơn vị đầu vào).
\begin{table}[h!]
    \centering
    \caption{Các bài toán dùng trong thực nghiệm huấn luyện mạng ANN khác cấu trúc}

	\begin{tabular}{|c|c|c|c|c|}
        \hline
        \multirow{1}{*}{\textbf{STT}} & 
        \multicolumn{1}{c|} {\textbf{Bài toán}} & \multicolumn{1}{c|}{Số đơn vị đầu vào} &  \multicolumn{1}{c|}{\textbf{Tổng số điểm dữ liệu}}\\ \hline
        1 & 4-bit & 4  & 16 \\\hline
        2 & 6-bit & 6  & 64 \\\hline
        3 & 8-bit & 8  & 256 \\\hline
        4 & 9-bit & 9  & 512 \\\hline
        5 & 10-bit & 10  & 1024 \\\hline
        6 & Breast cancer & 9  & 699 \\\hline
        7 & Tic-tac-toe & 9  & 958 \\\hline
        8 & Ionosphere & 34  & 351 \\\hline
        9 & Credit screening & 14  & 653 \\\hline

    \end{tabular}
    \label{tab:result:nbit}
\end{table}
\subsubsection{Cấu hình ANN thực nghiệm}
Thực nghiệm trong đồ án được xét dựa trên tiêu chí độ sâu của mạng ANN. Các bài toán với các mạng ANN có độ sâu giống nhau và khác nhau được cấu hình thực nghiệm với $K=3$. Thực nghiệm được chia thành 2 phần chính gồm:
\begin{itemize}
    \item Mạng sâu cùng số lớp: Độ sâu mạng $L = 1$ và $L = 2$ cho tất cả các tác vụ học, các cấu hình mạng khác nhau về số lượng đơn vị xử lý trên mỗi lớp. Thực nghiệm này nhắm đến đánh giá hiệu quả chia sẻ trọng số theo chiều rộng của mạng.
    \item Mạng sâu khác số lớp: Độ sâu mạng của mỗi tác vụ học là khác nhau, thay đổi từ $L = 3$, đến $L = 5$, số lượng đơn vị xử lý trên mỗi lớp đều giống nhau và được thiết lập bằng 2. Thực nghiệm này nhắm đến đánh giá hiệu quả chia sẻ trọng số theo chiều sâu của mạng.
\end{itemize}
Tổng kết lại ta có các bảng cấu hình thực nghiệm sau:    
    \begin{table}[h!]
        \centering
        \caption{Bộ dữ liệu huấn luyện ANN cùng độ sâu 1 lớp ẩn đơn giản}

    	\begin{tabular}{|c|c|c|c|c|}
            \hline
            \multirow{1}{*}{\textbf{Bài toán}} & 
            \multicolumn{1}{c|} {\textbf{Tên tác vụ}} & \multicolumn{1}{c|}{\textbf{Cấu trúc lần lượt của từng tác vụ}}\\ \hline
            
            \multirow{1}{*} 
            {4bit} 
            &  4bit (4-5-6) &  (4,4,1)-(4,5,1)-(4,6,1)\\\hline
            \multirow{2}{*} 
            {6bit} 
            &  6bit (5-6-7) & (6,5,1)-(6,6,1)-(6,7,1)\\ \cline{2-3}
            &  6bit (6-7-8) & (6,6,1)-(6,7,1)-(6,8,1)\\ \hline
            \multirow{2}{*} 
            {8bit} 
            &  8bit (5-6-7) & (8,5,1)-(8,6,1)-(8,7,1)\\\cline{2-3}
            &  8bit (6-7-8) & (8,6,1)-(8,7,1)-(8,8,1)\\\hline

        \end{tabular}
        \label{tab:result:nbit}
    \end{table}
    
    \begin{table}[h!]
        \centering
        \caption{Bộ dữ liệu huấn luyện nhiều mô ANN đa lớp}

    	\begin{tabular}{|c|c|c|c|c|}
            \hline
            \multirow{2}{*}{\textbf{Bài toán}} & 
            \multicolumn{2}{c|} {\textbf{ANN cùng độ sâu}} & \multicolumn{2}{c|}{\textbf{ANN khác độ sâu}}\\ \cline{2-5}
            &\multicolumn{1}{c|} {\textbf{Tên tác vụ}} & \multicolumn{1}{c|}{\textbf{Cấu trúc mạng}} & \multicolumn{1}{c|} {\textbf{Tên tác vụ}} & \multicolumn{1}{c|}{\textbf{Cấu trúc mạng}}\\ \hline
            
            \multirow{3}{*} 
            {8bit} &  Tác vụ 1 & (6,2) & Tác vụ 2 & (2,2,2,2,2) \\ \cline{2-5}
             & Tác vụ 2 & (6,3) & Tác vụ 2 & (2,2,2,2)\\ \cline{2-5}
            & Tác vụ 3 & (6,4) & Tác vụ 3 & (2,2,2)\\ \hline
            \multirow{3}{*} 
            {9bit} &  Tác vụ 1 & (6,2) & Tác vụ 1 & (2,2,2,2,2) \\ \cline{2-5}
             & Tác vụ 2 & (6,3) & Tác vụ 2 & (2,2,2,2)\\ \cline{2-5}
            & Tác vụ 3 & (6,4) & Tác vụ 3 & (2,2,2) \\ \hline
            \multirow{3}{*} 
            {10bit} &  Tác vụ 1 & (6,2) & Tác vụ 1 & (2,2,2,2,2) \\ \cline{2-5}
             & Tác vụ 2 & (6,3) & Tác vụ 2 & (2,2,2,2)\\ \cline{2-5}
            & Tác vụ 3 & (6,4) & Tác vụ 3 & (2,2,2)\\ \hline
        \end{tabular}
        \label{tab:result:nbit}
    \end{table}
    
    \begin{table}[h!]
        \centering
        \caption{Danh sách các bộ dữ liệu UCI cho huấn luyện ANN}
    	\begin{tabular}{|c|c|c|c|c|}
            \hline
            \multirow{2}{*}{\textbf{Bài toán}} & 
            \multicolumn{2}{c|} {\textbf{ANN cùng độ sâu}} & \multicolumn{2}{c|}{\textbf{ANN khác độ sâu}}\\ \cline{2-5}
            &\multicolumn{1}{c|} {\textbf{Tên tác vụ}} & \multicolumn{1}{c|}{\textbf{Cấu trúc mạng}} & \multicolumn{1}{c|} {\textbf{Tên tác vụ}} & \multicolumn{1}{c|}{\textbf{Cấu trúc mạng}}\\ \hline
            
            \multirow{3}{*} 
            {ionosphere} &  Tác vụ 1 & (5,2) & Tác vụ 2 & (2,2,2,2,2) \\ \cline{2-5}
             & Tác vụ 2 & (5,3) & Tác vụ 2 & (2,2,2,2)\\ \cline{2-5}
            & Tác vụ 3 & (5,4) & Tác vụ 3 & (2,2,2)\\ \hline
            \multirow{3}{*} 
            {ticTacToe} &  Tác vụ 1 & (5,2) & Tác vụ 1 & (2,2,2,2,2) \\ \cline{2-5}
             & Tác vụ 2 & (5,3) & Tác vụ 2 & (2,2,2,2)\\ \cline{2-5}
            & Tác vụ 3 & (5,4) & Tác vụ 3 & (2,2,2) \\ \hline
            \multirow{3}{*} 
            {creditScreening} &  Tác vụ 1 & (6,2) & Tác vụ 1 & (2,2,2,2,2) \\ \cline{2-5}
             & Tác vụ 2 & (6,3) & Tác vụ 2 & (2,2,2,2)\\ \cline{2-5}
            & Tác vụ 3 & (6,4) & Tác vụ 3 & (2,2,2)\\ \hline
            \multirow{3}{*} 
            {breastCancer} &  Tác vụ 1 & (6,2) & Tác vụ 1 & (2,2,2,2,2) \\ \cline{2-5}
             & Tác vụ 2 & (6,3) & Tác vụ 2 & (2,2,2,2)\\ \cline{2-5}
            & Tác vụ 3 & (6,4) & Tác vụ 3 & (2,2,2)\\ \hline
        \end{tabular}
        \label{tab:result:nbit}
    \end{table}

\subsection{Các môi trường học tăng cường}
\subsubsection{Acrobot}
\begin{figure}[h!]
    \centering
    \scalebox{0.3}{\fbox{\includegraphics{thesis/images/acrobot_env.png}}}
    \caption{Trò chơi Acrobot}
    \label{fig:flappybird}
\end{figure}
Acrobot là một con lắc 2 liên kết chỉ có khớp thứ hai được kích hoạt. Ban đầu, cả hai liên kết đều hướng xuống dưới. Mục tiêu là để xoay bộ phận đầu cuối ở độ cao mà ít nhất là chiều dài của một liên kết nằm ở phía trên đường cơ sở. Cả hai liên kết có thể xoay tự do và có thể đi qua nhau, tức là, chúng không va chạm khi chúng có cùng một góc. Trò chơi bao gồm các trạng thái của môi trường là các hàm sin() và cos() của 2 khớp góc quay và vận tốc góc khớp đó là [$\cos(\theta_1)$, $\sin(\theta_1)$, $\cos(\theta_2)$, $\sin(\theta_2)$, $\Dot{\theta_1}$, $\Dot{\theta_2}$]. Đối với liên kết đầu tiên, một góc 0 tương ứng với liên kết hướng xuống dưới. Góc của liên kết thứ hai liên quan đến góc của liên kết thứ nhất. Một góc 0 tương ứng với việc có cùng một góc giữa hai liên kết. Trạng thái [1, 0, 1, 0, ..., ...] có nghĩa là cả hai liên kết đều hướng xuống dưới. Các g
hành động cung cấp là áp dụng mô-men +1, 0 hoặc -1 trên khớp nối giữa hai liên kết con lắc. 

Chiều dài của mỗi liên kết được khởi tạo ban đầu gọi là $l_1$, $l_2$, cấu hình mạng ANN được sử dụng để học là (6,8,1). Trong đồ án này sẽ định nghĩa các tác vụ dựa theo sự khác nhau về độ dài của liên kết thứ 2:
\begin{table} [H]
    \begin{center}
    \caption{Danh sách các tác vụ thực nghiệm bài toán Acrobot}
    \scalebox{0.9}{\begin{tabular}{|c|c|c|c|c|c|}
    \hline
    \multirow{1}{*}{\textbf{Tham số}} & \multicolumn{1}{c|}{\textbf{Tác vụ 1}} & \multicolumn{1}{c|}{\textbf{Tác vụ 2}} & \multicolumn{1}{c|}{\textbf{Tác vụ 3}} & \multicolumn{1}{c|}{\textbf{Tác vụ 4}} & \multicolumn{1}{c|}{\textbf{Tác vụ 5}} \\ \hline
    $l_2$ & $l_2=0.8$ & $l_2=0.9$ & $l_2=1.0$ & $l_2=1.1$ & $l_2=1.2$ \\\hline
    \end{tabular}}
    \end{center}
    \label{tab:result:flappybird}
\end{table}

\subsubsection{PixelCopter}
\begin{figure}[h!]
    \centering
    \scalebox{0.5}{\fbox{\includegraphics{thesis/images/pixelcopter.png}}}
    \caption{Trò chơi PixelCopter}
    \label{fig:flappybird}
\end{figure}
Pixelcopter là trò chơi đòi hỏi người chơi phải vượt qua các vật cản bên trong một hang động. Đây là bản sao chép của trò chơi máy bay trực thăng nổi tiếng (thuật ngữ gốc: \emph{helicopter}) với người chơi chỉ là một đơn vị pixel khiêm tốn. 
Với mỗi khối đơn vị chiều dài mà người chơi vượt qua sẽ nhận được một điểm thưởng là +1. Khi trò chơi kết thúc sẽ nhận được một điểm thưởng âm là -1. Trò chơi kết thúc khi người choi va đập vào thành hang, hoặc các chướng ngại vật trong hang. Trạng thái của trò chơi được biểu diễn bởi một véc-tơ 7 chiều mô tả vị trí của pixel và các vật cản ở gần. Để mô hình hóa policy của bài toán dưới dạng mạng ANN, đồ án sử dụng một mạng ANN có cấu trúc là (7,8,1) tương ứng với đầu vào, lớp ẩn, đầu ra của policy.

Trong môi trường của trò chơi có một tham số là momentum ảnh hưởng đến tốc độ, và vị trí sau mỗi hành động của pixel. Trong đồ án này sẽ định nghĩa các tác vụ dựa theo sự khác nhau về momentum giữa các môi trường:
\begin{table} [H]
    \begin{center}
    \caption{Danh sách các tác vụ thực nghiệm bài toán Pixelcopter}

    \scalebox{0.85}{\begin{tabular}{|c|c|c|c|c|c|}
    \hline
    \multirow{1}{*}{\textbf{Tham số}} & \multicolumn{1}{c|}{\textbf{Tác vụ 1}} & \multicolumn{1}{c|}{\textbf{Tác vụ 2}} & \multicolumn{1}{c|}{\textbf{Tác vụ 3}} & \multicolumn{1}{c|}{\textbf{Tác vụ 4}} & \multicolumn{1}{c|}{\textbf{Tác vụ 5}} \\ \hline
    momentum & $momentum=0$ & $momentum=0.1$ & $momentum=0.2$ & $momentum=0.3$ & $momentum=0.4$\\\hline
    \end{tabular}}
    \end{center}
    \label{tab:result:flappybird}
\end{table}

\subsubsection{FlappyBird}
\begin{figure}[h!]
    \centering
    \scalebox{0.5}{\fbox{\includegraphics{flappy-bird_tbqj.jpg}}}
    \caption{Trò chơi FlappyBird}
    \label{fig:flappybird}
\end{figure}
Là trò chơi mà ở đó tác nhân (chú chim) phải vượt qua được khoảng trống giữa các ống. Trong trò chơi tác nhân chỉ thực hiện 2 hành động: hướng lên, hướng xuống. Mũi tên hướng lên khiến chim trong trò chơi sẽ đi lên, mũi trên hướng xuống khiến chim đi xuống. Trong trường hợp chim đập xuống đất, đập vào thành ống hoặc đập lên phía trên màn hình thì trò chơi sẽ kết thúc. Mỗi lượt chim qua một ống sẽ được tính là được thêm $+1$ điểm thưởng. Mỗi lần tới trạng thái kết thúc sẽ nhận được một điểm thưởng âm là $-1$. Có 8 trạng thái biểu diễn vị trí 2D của chim, vị trí của vật cản tiếp theo, vị trị của vật cản tiếp theo sau đó. Với bài toán này trong đồ án sẽ sử dụng một mạng ANN có cấu hình là $(8,4,1)$ tương ứng với đầu vào, lớp ẩn, đầu ra của mạng để có thể học được mô hình policy của bài toán.

Trong môi trường của trò chơi có một tham số là trọng lực (thuật ngữ gốc: \emph{gravity}. Trong đồ án này sẽ định nghĩa các tác vụ dựa theo sự khác nhau về trọng lực giữa các môi trường:
\begin{table} [H]
    \begin{center}
    \caption{Danh sách các tác vụ thực nghiệm bài toán FlappyBird}

    \scalebox{0.9}{\begin{tabular}{|c|c|c|c|c|c|}
    \hline
    \multirow{1}{*}{\textbf{Tham số}} & \multicolumn{1}{c|}{\textbf{Tác vụ 1}} & \multicolumn{1}{c|}{\textbf{Tác vụ 2}} & \multicolumn{1}{c|}{\textbf{Tác vụ 3}} & \multicolumn{1}{c|}{\textbf{Tác vụ 4}} & \multicolumn{1}{c|}{\textbf{Tác vụ 5}} \\ \hline
    gravity & $gravity=0.8$ & $gravity=1.8$ & $gravity=2.8$ & $gravity=3.8$ & $gravity=4.8$\\\hline
    \end{tabular}}
    \end{center}
    \label{tab:result:flappybird}
\end{table}

\section{Cài đặt thực nghiệm}
Các thực nghiệm được triển khai hoàn toàn trên hệ thống Ubuntu 18.04 LTS 64-bit với mô tả cấu hình như phía dưới:
\begin{itemize}
    \item CPU: Intel\textregistered Core\texttrademark i5-2430M CPU @ 2.40GHz × 2
    \item RAM: 6GB
    \item Ngôn ngữ lập trình: Python 3.6
    \item Mã nguồn: https://github.com/minhquang4334/mfeaii-ann-rl
\end{itemize}

\subsection{Cấu hình cho bài toán huấn luyện các mạng neural khác cấu trúc}
Các thuật toán đều chạy trên mỗi cấu hình thực nghiệm 30 lần. Độ đo hiệu quả giải thuật là MSE được thống kê trên cả bộ dữ liệu học và bộ dữ liệu kiểm định. Độ đo được thống kê trong 30 lần chạy để so sánh kết quả trung bình cùng độ lệch chuẩn.
\begin{table}[h!]
    \centering
    \caption{Cấu hình và tham số giải thuật đề xuất cho bài toán huấn luyện các mạng neural khác cấu trúc}

	\begin{tabular}{|l|c|c|c|c|}
        \hline
        \multirow{1}{*}{\textbf{Tham số}} & 
        \multicolumn{1}{c|} {\textbf{Ký hiệu}} & \multicolumn{1}{c|}{\textbf{Giá trị}}\\ \hline
        Kích thước quần thể đơn nhiệm & $N_k$ & 30\\
        Số tác vụ & $K$ & 3\\
        Kích thước quần thể đa nhiệm & $N$ & $N_k \cdot K$\\
        Số thế hệ tiến hóa & $T$ & 1000\\
        Chỉ số phân phối SBX & $\eta_c$ & 15\\
        Tỉ lệ đột biến PMU & $\sigma$ & 0.2\\
        Chỉ số đột biến PMU & $\eta_m$ & 15\\
        % pswap & - & 0.5\\
        Giá trị tham số cố định $rmp$ của thuật toán $MFEA$ & $rmp$ & 0.5\\
        Giá trị khởi tạo các phần tử trong ma trận $RMP$ & $rmp_{k,j}$ & 0\\
        Giá trị chặn dưới của từng phần tử $rmp_{kj}, k,j \in {K}$  & - & 0.1\\
        Số lần chạy thống kê & - & 30\\ \hline
    \end{tabular}
    \label{tab:config:nbit}
\end{table}
\subsection{Cấu hình cho bài toán huấn luyện nhiều mô hình học tăng cường}
Các thuật toán đều chạy trên mỗi cấu hình thực nghiệm 30 lần. Độ đo hiệu quả giải thuật là tổng phần thưởng thu được với bộ tham số hiện tại. Độ đo được thống kê trong 30 lần chạy để so sánh kết quả trung bình cùng độ lệch chuẩn.
\begin{table}[h!]
    \centering
    \caption{Cấu hình và tham số giải thuật đề xuất cho bài toán huấn luyện nhiều mô hình học tăng cường}

    \begin{tabular}{|l|c|c|c|c|}
        \hline
        \multirow{1}{*}{\textbf{Tham số}} & 
        \multicolumn{1}{c|} {\textbf{Ký hiệu}} & \multicolumn{1}{c|}{\textbf{Giá trị}}\\ \hline
        Kích thước quần thể đơn nhiệm & $N_k$ & 30\\
        Số tác vụ & $K$ & 5\\
        Kích thước quần thể đa nhiệm & $N$ & $N_k \cdot K$\\
        Số thế hệ tiến hóa & $T$ & 200\\
        Chỉ số phân phối SBX & $\eta_c$ & 15\\
        Tỉ lệ đột biến PMU & $\sigma$ & 0.2\\
        Chỉ số đột biến PMU & $\eta_m$ & 15\\
        % pswap & - & 0.5\\
        Giá trị tham số cố định $rmp$ của thuật toán $MFEA$ & $rmp$ & 0.5\\
        Giá trị khởi tạo các phần tử trong ma trận $RMP$ & $rmp_{k,j}$ & 0\\
        Giá trị chặn dưới của từng phần tử $rmp_{kj}, k,j \in {K}$  & - & 0.1\\
        Số lần chạy thống kê & - & 30\\ \hline
        
    \end{tabular}
    \label{tab:config:rl}
\end{table}


\section{Kết quả}
	\begin{frame}{Huấn luyện mạng nơ-ron đa lớp - Bộ dữ liệu thực nghiệm}
	   \begin{table}
        \centering
    	\begin{tabular}{|c|c|c|c|c|}
            \hline
            {\textbf{STT}} & 
            \multicolumn{1}{c|} {\textbf{Bài toán}} & \multicolumn{1}{c|}{\textbf{Số đơn vị đầu vào}} &  \multicolumn{1}{c|}{\textbf{Tổng số điểm dữ liệu}}\\ \hline
            1 & 4-bit & 4  & 16 \\\hline
            2 & 6-bit & 6  & 64 \\\hline
            3 & 8-bit & 8  & 256 \\\hline
            4 & 9-bit & 9  & 512 \\\hline
            5 & 10-bit & 10  & 1024 \\\hline
            6 & Breast cancer & 9  & 699 \\\hline
            7 & Tic-tac-toe & 9  & 958 \\\hline
            8 & Ionosphere & 34  & 351 \\\hline
            9 & Credit screening & 14  & 653 \\\hline
        \end{tabular}
        \label{tab:result:nbit}
        \caption{Các bài toán phân loại nhị phân dùng trong thực nghiệm}
    \end{table}
	\end{frame}
	\begin{frame}{Thực nghiệm}
	    \begin{table}[h!]
        	\begin{tabular}{|l|c|c|}
                \hline
                \multirow{1}{*}{\textbf{Tham số}} & 
                \multicolumn{1}{c|} {\textbf{Ký hiệu}} & \multicolumn{1}{c|}{\textbf{Giá trị}}\\ \hline
                Kích thước quần thể đơn nhiệm &  & 30\\
                Kích thước quần thể đa nhiệm &  & 30\\
                Số thế hệ tiến hóa &  & 1000\\
                Chỉ số phân phối SBX &  & 15\\
                Tỉ lệ đột biến PMU &  & 0.2\\
                Chỉ số đột biến PMU &  & 15\\
                pswap & & 0.5\\
                Giá trị tham số cố định $rmp$ của thuật toán $MFEA$ &  & 0.5\\
                Giá trị khởi tạo các phần tử trong ma trận $RMP$ &  & 0\\
                Giá trị chặn dưới của từng phần tử $rmp_{kj}, k,j \in {K}$  &  & 0.1\\
                Số lần chạy thống kê & & 30\\ \hline
                
        
            \end{tabular}
            \label{tab:config:nbit}
            \caption{Cấu hình và tham số giải thuật đề xuất cho bài toán huấn luyện các mạng nơ-ron đa lớp}
        \end{table}
	\end{frame}
	\begin{frame}{Cấu hình tác vụ}
	    \begin{table}[h!]
        \centering
    	\begin{tabular}{|c|c|c|c|c|}
            \hline
            \multirow{1}{*}{\textbf{Bài toán}} & 
            \multicolumn{1}{c|} {\textbf{Tên tác vụ}} & \multicolumn{1}{c|}{\textbf{Cấu trúc lần lượt của từng tác vụ}}\\ \hline
            
            \multirow{1}{*}
            {4bit} 
            &  4bit (4-5-6) &  (4,4,1)-(4,5,1)-(4,6,1)\\\hline
            \multirow{2}{*} 
            {6bit} 
            &  6bit (5-6-7) & (6,5,1)-(6,6,1)-(6,7,1)\\ \cline{2-3}
            &  6bit (6-7-8) & (6,6,1)-(6,7,1)-(6,8,1)\\ \hline
            \multirow{2}{*} 
            {8bit} 
            &  8bit (5-6-7) & (8,5,1)-(8,6,1)-(8,7,1)\\\cline{2-3}
            &  8bit (6-7-8) & (8,6,1)-(8,7,1)-(8,8,1)\\\hline

        \end{tabular}
        \label{tab:result:nbit}
        \caption{Bài toán huấn luyện mạng nơ-ron 1 lớp ẩn đơn giản}
    \end{table}
    \begin{table}[h!]
        \centering
    	\begin{tabular}{|c|c|c|c|c|}
            \hline
            \multirow{2}{*}{\textbf{Bài toán}} & 
            \multicolumn{2}{c|} {\textbf{ANN cùng độ sâu}} & \multicolumn{2}{c|}{\textbf{ANN khác độ sâu}}\\ \cline{2-5}
            &\multicolumn{1}{c|} {\textbf{Tên tác vụ}} & \multicolumn{1}{c|}{\textbf{Cấu trúc mạng}} & \multicolumn{1}{c|} {\textbf{Tên tác vụ}} & \multicolumn{1}{c|}{\textbf{Cấu trúc mạng}}\\ \hline
            
            \multirow{3}{*} 
            {8bit} &  Tác vụ 1 & (6,2) & Tác vụ 1 & (2,2,2,2,2) \\ \cline{2-5}
             & Tác vụ 2 & (6,3) & Tác vụ 2 & (2,2,2,2)\\ \cline{2-5}
            & Tác vụ 3 & (6,4) & Tác vụ 3 & (2,2,2)\\ \hline
            \multirow{3}{*} 
            {9bit} &  Tác vụ 1 & (6,2) & Tác vụ 1 & (2,2,2,2,2) \\ \cline{2-5}
             & Tác vụ 2 & (6,3) & Tác vụ 2 & (2,2,2,2)\\ \cline{2-5}
            & Tác vụ 3 & (6,4) & Tác vụ 3 & (2,2,2) \\ \hline
            \multirow{3}{*} 
            {10bit} &  Tác vụ 1 & (6,2) & Tác vụ 1 & (2,2,2,2,2) \\ \cline{2-5}
             & Tác vụ 2 & (6,3) & Tác vụ 2 & (2,2,2,2)\\ \cline{2-5}
            & Tác vụ 3 & (6,4) & Tác vụ 3 & (2,2,2)\\ \hline
        \end{tabular}
        \label{tab:result:nbit}
        \caption{Bộ dữ liệu huấn luyện nhiều mô mạng Nơ-ron đa lớp}
    \end{table}
	\end{frame}
	
	\begin{frame}{Cấu hình tác vụ}
    \begin{table}[h!]
        \centering
    	\begin{tabular}{|c|c|c|c|c|}
            \hline
            \multirow{2}{*}{\textbf{Bài toán}} & 
            \multicolumn{2}{c|} {\textbf{ANN cùng độ sâu}} & \multicolumn{2}{c|}{\textbf{ANN khác độ sâu}}\\ \cline{2-5}
            &\multicolumn{1}{c|} {\textbf{Tên tác vụ}} & \multicolumn{1}{c|}{\textbf{Cấu trúc mạng}} & \multicolumn{1}{c|} {\textbf{Tên tác vụ}} & \multicolumn{1}{c|}{\textbf{Cấu trúc mạng}}\\ \hline
            
            \multirow{3}{*} 
            {ionosphere} &  Tác vụ 1 & (5,2) & Tác vụ 1 & (2,2,2,2,2) \\ \cline{2-5}
             & Tác vụ 2 & (5,3) & Tác vụ 2 & (2,2,2,2)\\ \cline{2-5}
            & Tác vụ 3 & (5,4) & Tác vụ 3 & (2,2,2)\\ \hline
            \multirow{3}{*} 
            {ticTacToe} &  Tác vụ 1 & (5,2) & Tác vụ 1 & (2,2,2,2,2) \\ \cline{2-5}
             & Tác vụ 2 & (5,3) & Tác vụ 2 & (2,2,2,2)\\ \cline{2-5}
            & Tác vụ 3 & (5,4) & Tác vụ 3 & (2,2,2) \\ \hline
            \multirow{3}{*} 
            {creditScreening} &  Tác vụ 1 & (6,2) & Tác vụ 1 & (2,2,2,2,2) \\ \cline{2-5}
             & Tác vụ 2 & (6,3) & Tác vụ 2 & (2,2,2,2)\\ \cline{2-5}
            & Tác vụ 3 & (6,4) & Tác vụ 3 & (2,2,2)\\ \hline
            \multirow{3}{*} 
            {breastCancer} &  Tác vụ 1 & (6,2) & Tác vụ 1 & (2,2,2,2,2) \\ \cline{2-5}
             & Tác vụ 2 & (6,3) & Tác vụ 2 & (2,2,2,2)\\ \cline{2-5}
            & Tác vụ 3 & (6,4) & Tác vụ 3 & (2,2,2)\\ \hline
        \end{tabular}
        \label{tab:result:nbit}
        \caption{Bộ dữ liệu huấn luyện UCI}
    \end{table}
	\end{frame}
	\begin{frame}{6-bit even problem}
		\begin{table} [H]
        	\caption{Bài 4 bit 1 lớp ẩn}
            \begin{center}
            \begin{tabular}{|c|c|c|c|}
            \hline
            \multirow{1}{*}{\textbf{Method}} & \multicolumn{1}{c|}{\textbf{Subtask1}} & \multicolumn{1}{c|}{\textbf{Subtask 2}} & \multicolumn{1}{c|}{\textbf{Subtask 3}} \\ \hline
            CEA & $0.0316 \pm 0.0125$ & $0.0201 \pm 0.011922$ & $0.0117 \pm 0.008133$ \\
            MFEA-I & $0.025 \pm 0.012957$ & $0.0117 \pm 0.007342$ & $0.0072 \pm 0.005355$\\
            MFEA-II  & $\mathbf{0.0219 \pm 0.009181}$ & $\mathbf{0.0099 \pm 0.007053}$ & $\mathbf{0.0052 \pm 0.004126}$ \\\hline
            
            \end{tabular}
            \end{center}
            
            \label{tab:result:nbit}
            \caption{6-bit even parity problem}
            \begin{center}
            \begin{tabular}{|c|c|c|c|}
            \hline
            \multirow{1}{*}{\textbf{Method}} & \multicolumn{1}{c|}{\textbf{Subtask1}} & \multicolumn{1}{c|}{\textbf{Subtask 2}} & \multicolumn{1}{c|}{\textbf{Subtask 3}} \\ \hline
            CEA(5,6,7)  & $0.0703 \pm 0.014543$ & $0.0619 \pm 0.018078$ & $0.0572 \pm 0.017982$ \\
            MFEA-I(5,6,7)   & $\mathbf{0.06 \pm 0.014702}$ & $\mathbf{0.0498 \pm 0.009562}$ & $0.047 \pm 0.008464$ \\
            MFEA-II(5,6,7)  & $0.06 \pm 0.011387$ & $0.052 \pm 0.009393$ & $\mathbf{0.047 \pm 0.009579}$ \\\hline
            
            CEA(6,7,8)   & $0.0669 \pm 0.016032$ & $0.0583 \pm 0.009948$ & $0.0521 \pm 0.015598$ \\
            MFEA-I(6,7,8)  & $0.0611 \pm 0.013622$ & $0.0528 \pm 0.011122$ & $0.0484 \pm 0.011074$ \\
            MFEA-II(6,7,8) & $\mathbf{0.0522 \pm 0.011066}$ & $\mathbf{0.0476 \pm 0.01033}$ & $\mathbf{0.0418 \pm 0.011741}$ \\\hline
            
            \end{tabular}
            \end{center}
        \end{table}
	\end{frame}
	
    \begin{frame}{8-bit even problem}
    \begin{table} [H]
        \caption{8-bit even parity problem}
        \begin{center}
        \begin{tabular}{|c|c|c|c|}
        \hline
        \multirow{1}{*}{\textbf{Method}} & \multicolumn{1}{c|}{\textbf{Subtask1}} & \multicolumn{1}{c|}{\textbf{Subtask 2}} & \multicolumn{1}{c|}{\textbf{Subtask 3}} \\ \hline
        CEA(5,6,7) & $0.0956 \pm 0.013764$ & $0.0906 \pm 0.015274$ & $0.0874 \pm 0.010884$ \\
        MFEA-I(5,6,7) & $0.0859 \pm 0.011722$ & $0.0801 \pm 0.009489$ & $0.0778 \pm 0.010507$  \\
        MFEA-II(5,6,7) & $\mathbf{0.0827 \pm 0.010678}$ & $\mathbf{0.076 \pm 0.012979}$ & $\mathbf{0.0735 \pm 0.012598}$ \\\hline
        
        CEA(6,7,8)& $0.0895 \pm 0.012499$ & $0.0902 \pm 0.013171$ & $0.081 \pm 0.013371$ \\
        MFEA-I(6,7,8)  & $0.0826 \pm 0.011089$ & $0.0768 \pm 0.010228$ & $0.0734 \pm 0.008805$ \\
        MFEA-II(6,7,8) & $\mathbf{0.0808 \pm 0.010726}$ & $\mathbf{0.0739 \pm 0.01117}$ & $\mathbf{0.072 \pm 0.009657}$ \\\hline
        \end{tabular}
        \end{center}
        
        \label{tab:result:nbit}
    \end{table}
    \end{frame}
    \begin{frame}{Bài 4bit}
        \begin{figure}[H]
            \centering
            \scalebox{.6}{\includegraphics[width=\textwidth,height=\textheight,keepaspectratio]{images/results/nbit_1layer/4bit_task.png}}
            \scalebox{.6}{\includegraphics[width=\textwidth,height=\textheight,keepaspectratio]{images/results/nbit_1layer/4bit_rmp.png}}
            \caption{Bài 4bit: Biểu đồ tương quan giá trị rmp giữa các cặp tác vụ và hội tụ của từng tác vụ với MFEA2}
            \label{fig:my_label}
        \end{figure}
    \end{frame}
    \begin{frame}{Bài 6bit}
        \begin{columns}
            \column{0.5\textwidth}
            \begin{figure}[H]
                \centering
                \scalebox{.9}{\includegraphics[width=\textwidth,height=\textheight,keepaspectratio]{images/results/nbit_1layer/6bit1_task.png}}
                \scalebox{.9}{\includegraphics[width=\textwidth,height=\textheight,keepaspectratio]{images/results/nbit_1layer/6bit1_rmp.png}}
                \caption{Bài 6bit(5,6,7): Biểu đồ tương quan giá trị rmp giữa các cặp tác vụ và hội tụ của từng tác vụ với MFEA2}
                \label{fig:my_label}
            \end{figure}
            \column{0.5\textwidth}
            \begin{figure}[H]
                \centering
                \scalebox{.9}{\includegraphics[width=\textwidth,height=\textheight,keepaspectratio]{images/results/nbit_1layer/6bit2_task.png}}
                \scalebox{.9}{\includegraphics[width=\textwidth,height=\textheight,keepaspectratio]{images/results/nbit_1layer/6bit2_rmp.png}}
                \caption{Bài 6bit(6,7,8): Biểu đồ tương quan giá trị rmp giữa các cặp tác vụ và hội tụ của từng tác vụ với MFEA2}
                \label{fig:my_label}
            \end{figure}
        \end{columns}
    \end{frame}
    \begin{frame}{Bài 8bit}
        \begin{columns}
            \column{0.5\textwidth}
            \begin{figure}[H]
                \centering
                \scalebox{.9}{\includegraphics[width=\textwidth,height=\textheight,keepaspectratio]{images/results/nbit_1layer/8bit1_task.png}}
                \scalebox{.9}{\includegraphics[width=\textwidth,height=\textheight,keepaspectratio]{images/results/nbit_1layer/8bit1_rmp.png}}
                \caption{Bài 8bit(5,6,7): Biểu đồ tương quan giá trị rmp giữa các cặp tác vụ và hội tụ của từng tác vụ với MFEA2}
                \label{fig:my_label}
            \end{figure}
            \column{0.5\textwidth}
            \begin{figure}[H]
            \centering
            \scalebox{.9}{\includegraphics[width=\textwidth,height=\textheight,keepaspectratio]{images/results/nbit_1layer/8bit2_task.png}}
            \scalebox{.9}{\includegraphics[width=\textwidth,height=\textheight,keepaspectratio]{images/results/nbit_1layer/8bit2_rmp.png}}
            \caption{Bài 8bit(6,7,8): Biểu đồ tương quan giá trị rmp giữa các cặp tác vụ và hội tụ của từng tác vụ với MFEA2}
            \label{fig:my_label}
        \end{figure}
        \end{columns}
    \end{frame}
    
    
    \begin{frame}{Mạng ANN 2 lớp ẩn}
        \begin{table} [H]
        \caption{Thực nghiệm kết quả mạng ANN 2 lớp ẩn}
        \begin{center}
        \begin{tabular}{|c|c|c|c|c|}
        \hline
        \multirow{1}{*}{\textbf{Bài toán}} &
        \multirow{1}{*}{\textbf{Method}} & \multicolumn{1}{c|}{\textbf{Tác vụ 1}} & \multicolumn{1}{c|}{\textbf{Tác vụ 2}} & \multicolumn{1}{c|}{\textbf{Tác vụ 3}} \\ \hline
        \multirow{3}{*} 
        {8-bit} &
        CEA & $0.075 \pm 0.012865$ & $0.0713 \pm 0.013116$ & $0.0718 \pm 0.012432$  \\
        & MFEA-I & $0.0738 \pm 0.012508$ & $0.0684 \pm 0.013252$ & $0.0669 \pm 0.015076$   \\
        & MFEA-II & $\mathbf{0.0705 \pm 0.012856}$ & $\mathbf{0.0624 \pm 0.011079}$ & $\mathbf{0.0595 \pm 0.011252}$\\\hline
        \multirow{3}{*} 
        {9-bit} &
        CEA & $0.0826 \pm 0.010588$ & $0.0751 \pm 0.014406$ & $0.0785 \pm 0.010766$  \\
        & MFEA-I & $0.0827 \pm 0.010438$ & $0.0762 \pm 0.009495$ & $0.0737 \pm 0.009766$ \\
        & MFEA-II & $\mathbf{0.0795 \pm 0.012865}$ & $\mathbf{0.0705 \pm 0.009581}$ & $\mathbf{0.0685 \pm 0.011156}$ \\\hline
        \multirow{3}{*} 
        {10-bit} &
        CEA & $0.0853 \pm 0.01105$ & $0.0862 \pm 0.008326$ & $0.0856 \pm 0.008919$  \\
        & MFEA-I  & $0.0855 \pm 0.012229$ & $\mathbf{0.0782 \pm 0.009659}$ & $\mathbf{0.0752 \pm 0.009681}$ \\
        & MFEA-II & $\mathbf{0.0833 \pm 0.010222}$ & $0.0791 \pm 0.009846$ & $0.0762 \pm 0.010419$ \\\hline
        \end{tabular}
        \end{center}
        
        \label{tab:result:nbit}
        \end{table}
    \end{frame}
    \begin{frame}{Mạng nơ-ron khác độ sâu}
        \begin{table} [H]
    \caption{Mạng ANN khác độ sâu}
    \begin{center}
    \begin{tabular}{|c|c|c|c|c|}
    \hline
    \multirow{1}{*}{\textbf{Bài toán}} &
    \multirow{1}{*}{\textbf{Method}} & \multicolumn{1}{c|}{\textbf{Tác vụ 1}} & \multicolumn{1}{c|}{\textbf{Tác vụ 2}} & \multicolumn{1}{c|}{\textbf{Tác vụ 3}} \\ \hline
    \multirow{3}{*} 
    {8-bit} &
    CEA & $0.0843 \pm 0.016042$ & $0.0842 \pm 0.014995$ & $0.0892 \pm 0.007818$ \\
    & MFEA-I & $\mathbf{0.0775 \pm 0.014683}$ & $0.0797 \pm 0.009343$ & $0.0812 \pm 0.017086$  \\
    & MFEA-II & $0.0837 \pm 0.021442$ & $\mathbf{0.0732 \pm 0.014932}$ & $\mathbf{0.0738 \pm 0.013425}$\\\hline
    \multirow{3}{*} 
    {9-bit} &
    CEA & $0.0916 \pm 0.009395$ & $0.0869 \pm 0.010216$ & $\mathbf{0.0786 \pm 0.010333}$ \\
    & MFEA-I  & $0.0858 \pm 0.01389$ & $0.0841 \pm 0.016001$ & $0.0794 \pm 0.015452$ \\
    & MFEA-II & $\mathbf{0.0845 \pm 0.009553}$ & $\mathbf{0.0827 \pm 0.010715}$ & $0.0889 \pm 0.011707$ \\\hline
    \multirow{3}{*} 
    {10-bit} &
    CEA & $0.0905 \pm 0.006423$ & $0.0901 \pm 0.015163$ & $0.0918 \pm 0.008761$  \\
    & MFEA-I  & $0.0917 \pm 0.006814$ & $0.0844 \pm 0.013841$ & $\mathbf{0.0861 \pm 0.009332}$ \\
    & MFEA-II & $\mathbf{0.0816 \pm 0.009981}$ & $\mathbf{0.0815 \pm 0.012087}$ & $0.0871 \pm 0.009619$ \\\hline
    \end{tabular}
    \end{center}
    
    \label{tab:result:nbit}
\end{table}

    \end{frame}
    \begin{frame}{Biểu đồ hội tụ bài cùng độ sâu}
        \begin{columns}
            \column{0.5\textwidth}
            \begin{figure}[H]
                \centering
                \scalebox{.9}{\includegraphics[width=\textwidth,height=\textheight,keepaspectratio]{images/results/nbit_2layer/8bit1_task.png}}
                \scalebox{.9}{\includegraphics[width=\textwidth,height=\textheight,keepaspectratio]{images/results/nbit_2layer/8bit1_rmp.png}}
                \caption{Bài 8-bit: Biểu đồ tương quan giá trị rmp giữa các cặp tác vụ và hội tụ của từng tác vụ với MFEA2}
                \label{fig:my_label}
            \end{figure}
            \column{0.5\textwidth}
            \begin{figure}[H]
            \centering
            \scalebox{.9}{\includegraphics[width=\textwidth,height=\textheight,keepaspectratio]{images/results/nbit_2layer/10bit1_task.png}}
            \scalebox{.9}{\includegraphics[width=\textwidth,height=\textheight,keepaspectratio]{images/results/nbit_2layer/10bit1_rmp.png}}
            \caption{Bài 10-bit: Biểu đồ tương quan giá trị rmp giữa các cặp tác vụ và hội tụ của từng tác vụ với MFEA2}
            \label{fig:my_label}
        \end{figure}
        \end{columns}
    \end{frame}
    \begin{frame}{UCI cùng đo}
    \begin{table}[h!]
        \begin{tabular}{|c|c|c|c|c|}
        \hline
        \multirow{1}{*}{\textbf{Instance}} & \multicolumn{1}{c|} {\textbf{Method}} & \multicolumn{1}{c|}{\textbf{Subtask1}} & \multicolumn{1}{c|}{\textbf{Subtask 2}} & \multicolumn{1}{c|}{\textbf{Subtask 3}} \\ \hline
        \multirow{3}{*} 
        {breastCancer} & CEA & $0.0097 \pm 0.0012$ & $0.0092 \pm 0.0007$ & $0.0093 \pm 0.0009$ \\
         & MFEA-I & $0.0097 \pm 0.0006$ & $0.0093 \pm 0.0005$ & $0.0091 \pm 0.0005$ \\ 
        & MFEA II & $\mathbf{0.0094 \pm 0.0008}$ & $\mathbf{0.0089 \pm 0.0006}$ & $\mathbf{0.0087 \pm 0.0004}$ \\ \hline
        \multirow{3}{*} {creditScreening} & CEA & $0.0509 \pm 0.0033$ & $0.0514 \pm 0.0033$ & $0.0508 \pm 0.004$ \\
       & MFEA-I & $0.0504 \pm 0.0024$ & $0.0503 \pm 0.0025$ & $0.0513 \pm 0.0022$ \\ 
       & MFEA-II & $\mathbf{0.0492 \pm 0.0023}$ & $\mathbf{0.0489 \pm 0.002}$ & $\mathbf{0.0491 \pm 0.002}$ \\ \hline
        \multirow{3}{*} {ionosphere} & CEA & $0.0384 \pm 0.0072$ & $0.0389 \pm 0.0129$ & $0.035 \pm 0.0047$ \\
        &MFEA-I & $0.0367 \pm 0.0068$ & $0.0347 \pm 0.0075$ & $0.0351 \pm 0.0088$ \\
        &MFEA-II & $\mathbf{0.0343 \pm 0.0079}$ & $\mathbf{0.0322 \pm 0.0071}$ & $\mathbf{0.032 \pm 0.007}$ \\\hline
        \multirow{3}{*} {ticTacToe} & CEA & $0.089 \pm 0.0047$ & $0.0838 \pm 0.0054$ & $0.0869 \pm 0.0049$ \\
        &MFEA-I & $0.0845 \pm 0.0049$ & $0.0818 \pm 0.0047$ & $0.0824 \pm 0.0047$ \\
        &MFEA-II & $\mathbf{0.082 \pm 0.0048}$ & $\mathbf{0.0815 \pm 0.0046}$ & $\mathbf{0.0812 \pm 0.0043}$  \\\hline
        
        \end{tabular}
    
        \label{tab:result:nbit}
        \caption{Huấn luyện nhiều ANN trên bộ dữ liệu UCI cùng độ sâu}
    \end{table}
    \end{frame}
    
    \begin{frame}{Khác độ sâu UCI}
        \begin{table}[h!]
        \begin{tabular}{|c|c|c|c|c|}
        \hline
        \multirow{1}{*}{\textbf{Instance}} & \multicolumn{1}{c|} {\textbf{Method}} & \multicolumn{1}{c|}{\textbf{Subtask1}} & \multicolumn{1}{c|}{\textbf{Subtask 2}} & \multicolumn{1}{c|}{\textbf{Subtask 3}} \\ \hline
        \multirow{3}{*} 
        {breastCancer} & CEA & $0.0119 \pm 0.0029$ & $0.0107 \pm 0.002$ & $\mathbf{0.0093 \pm 0.0005}$ \\
         & MFEA-I & $\mathbf{0.011 \pm 0.0015}$ & $0.0102 \pm 0.0012$ & $0.0094 \pm 0.0005$  \\ 
        & MFEA II & $\mathbf{0.011 \pm 0.0015}$ & $\mathbf{0.01 \pm 0.0011}$ & $0.0096 \pm 0.0011$ \\ \hline
        
        \multirow{3}{*} {creditScreening} & CEA & $0.054 \pm 0.0071$ & $0.0503 \pm 0.0027$ & $0.0485 \pm 0.0017$ \\
       & MFEA-I & $\mathbf{0.0508 \pm 0.0023}$ & $0.0497 \pm 0.0019$ & $0.0496 \pm 0.0021$ \\ 
       & MFEA-II & $0.0515 \pm 0.0033$ & $\mathbf{0.0494 \pm 0.0023}$ & $\mathbf{0.0485 \pm 0.0018}$ \\ \hline
       
        \multirow{3}{*} {ionosphere} & CEA & $0.0516 \pm 0.0124$ & $0.0449 \pm 0.0114$ & $0.0366 \pm 0.0094$ \\
        &MFEA-I & $0.049 \pm 0.0131$ & $0.0413 \pm 0.0112$ & $\mathbf{0.0365 \pm 0.007}$ \\
        &MFEA-II & $\mathbf{0.0473 \pm 0.0083}$ & $\mathbf{0.0387 \pm 0.0109}$ & $0.0367 \pm 0.008$  \\\hline
        
        \multirow{3}{*} {ticTacToe} & CEA & $0.0899 \pm 0.0067$ & $0.0879 \pm 0.0079$ & $0.0852 \pm 0.0045$  \\
        &MFEA-I & $\mathbf{0.088 \pm 0.008}$ & $0.0886 \pm 0.0064$ & $0.0843 \pm 0.0092$  \\
        &MFEA-II & $\mathbf{0.088 \pm 0.0073}$ & $\mathbf{0.0832 \pm 0.0067}$ & $0\mathbf{.0817 \pm 0.0046}$  \\\hline
        
        \end{tabular}
    
        \label{tab:result:nbit}
        \caption{Huấn luyện nhiều ANN trên bộ dữ liệu UCI khác độ sâu}
    \end{table}
    \end{frame}
    
    \begin{frame}{Học tăng cường}
    \begin{table}
    \caption{Mô hình học tăng cường khác môi trường}
    \begin{center}
    \begin{tabular}{|c|c|c|c|c|c|}
    \hline
    \multirow{1}{*}{\textbf{Trọng lực}} &
    \multirow{1}{*}{\textbf{Method}} & \multicolumn{1}{c|}{\textbf{Cao nhất}} & {\textbf{Thấp nhất}} & \multicolumn{1}{c|}{\textbf{Trung Bình}} & \multicolumn{1}{c|}{\textbf{Độ lệch}} \\ \hline
    \multirow{3}{*} 
    {Trọng lực=1.0} &
    CEA & $71$ & $4$ & $28.86$ & $17.16$ \\
    & MFEA-I & $\mathbf{129}$ & $17$ &$46.71$ & $20.36$  \\
    & MFEA-II & $84$ & $15$ & $\mathbf{52.81}$  &$15.45$\\\hline
    \multirow{3}{*} 
    {Trọng lực=1.98} &
    CEA & $382$ & $7$ &$174.86$ & $113.79$ \\
    & MFEA-I  & $\mathbf{546}$ & $98$ & $\mathbf{319.86}$ & $95.34$ \\
    & MFEA-II & $517$ & $\mathbf{131}$ &$311.38$ & $106.76$ \\\hline
    \multirow{3}{*} 
    {Trọng lực=2.96} &
    CEA & $272$ & $1$ &$97.9$ & $83.07$ \\
    & MFEA-I  & $\mathbf{349}$ & $\mathbf{140}$ & $\mathbf{222.81}$ & $52.07$ \\
    & MFEA-II & $322$ & $103$ &$212.38$ & $52.74$ \\\hline
    \multirow{3}{*} 
    {Trọng lực=3.94} &
    CEA & $227$ & $2$ &$69.24$ & $56.64$ \\
    & MFEA-I  & $\mathbf{217}$ & $\mathbf{95}$ & $\mathbf{142.86}$ & $27.4$ \\
    & MFEA-II & $190$ & $76$ &$140.38$ & $27.78$ \\\hline
    \multirow{3}{*} 
    {Trọng lực=4.92} &
    CEA & $95$ & $2$ & $25.48$ & $25.78$  \\
    & MFEA-I  & $\mathbf{181}$ & $\mathbf{54}$ & $90.86$ & $21.98$ \\
    & MFEA-II & $145$ & $44$ & $\mathbf{102.14}$ & $26.66$ \\\hline
    \end{tabular}
    \end{center}
    
    \label{tab:result:nbit}
\end{table}

    \end{frame}
    \begin{frame}{So sánh mức độ phân bố kết quả}
        \begin{figure}[h!]
        \centering
        \includegraphics[width=\textwidth,height=\textheight,keepaspectratio]{images/flappybird.png}
        \caption{Flappy Bird}
        \label{fig:FLP}
    \end{figure}
    \end{frame}
    
    \begin{frame}{So sánh biểu độ hội tụ kết quả}
        \begin{figure}[h!]
        \centering
        \includegraphics[width=\textwidth,height=\textheight,keepaspectratio]{images/flappybird_conv.png}
        \caption{Flappy Bird}
        \label{fig:FLP}
    \end{figure}
    \end{frame}
    \begin{frame}{So sánh biểu độ hội tụ kết quả Acrobot}
        \begin{figure}[h!]
        \centering
        \includegraphics[width=\textwidth,height=\textheight,keepaspectratio]{images/acobot_conv.png}
        \caption{Acrobot}
        \label{fig:Acrobot}
    \end{figure}
    \end{frame}
\pagebreak
\section{Nhận xét và bàn luận}
\subsection{Bài toán huấn luyện nhiều ANN khác cấu trúc}

Kết quả thực nghiệm chỉ rõ ưu điểm của cách tiếp cận của tiến hóa đa nhiệm trên tất cả các bộ dữ liệu thực nghiệm. Giải thuật CEA thể hiện không tốt trên tất cả các bài \emph{n-bit} và các bộ dữ liệu thực tế so với cách tiếp cận bằng tiến hóa đa nhiệm. Điều này có thể lý giải bởi với các bài toán có cấu trúc gần tương đồng với nhau việc trao đổi tri thức giữa các tác vụ giúp tăng tốc độ tối ưu trên từng tác vụ. Và khi so sánh giữa các giải thuật tiến hóa đa nhiệm với nhau thì MFEA-II thể hiện sự vượt trội so với MFEA-I nhờ ưu thế có thể xác định được mức độ chia sẻ tri thức phù hợp tại từng thời điểm. Các tác vụ được tối ưu hóa bởi MFEA-II có tốc độ nhanh hơn so với các thuật toán khác trên tất cả các tác vụ.

\subsubsection{Mạng cùng độ sâu}
Trong bài thực nghiệm ANN cùng độ sâu thuật toán MFEA-II tốt hơn so với CEA $100\%$ (8/8) số bộ dữ liệu, so với MFEA-I là $75\%$ số bộ dữ liệu (6/8). Với các bộ dữ liệu MFEA-II trội hơn thì xét về mức độ cải tiến, so với CEA là $10 \sim 12\%$, so với MFEA là $6 \sim 8\%$. Trên một số bài như trong hình \ref{fig:10bit_2layer} thì MFEA-II và MFEA-I có kết quả tương dương nhau tất cả các tác vụ. Có thể lý giải rằng với những bộ này mức độ tương đồng trung bình giữa các tác vụ là lớn việc dẫn tới việc tối ưu bằng MFEA-II hay MFEA-I không có nhiều khác biệt. Tuy nhiên MFEA-II vẫn có thể coi là tốt hơn nếu nhìn trên góc độ tốc độ hội tụ của thuật toán.

\subsubsection{Mạng khác độ sâu}
Trong bài thực nghiệm ANN cùng khác sâu thuật toán MFEA-II tốt hơn so với CEA và MFEA-I $100\%$ (3/3) số bộ dữ liệu. Xét về mức độ cải tiến, so với CEA là $12 \sim 15\%$, so với MFEA là $5 \sim 7\%$. Cá biệt trong hình \ref{fig:8bit_multilayer} thì mặc dù tính số lượng trên 3 tác vụ MFEA-II đem lại kết quả tốt hơn nhưng với tác vụ thứ 3, MFEA-II có kết quả kém hơn so với MFEA-I và CEA thông thường. Điều này có thể lý giải bởi ở các mạng sâu khác nhau về số lớp, lượng tham số chia sẻ chung giữa các cấu hình là thấp hơn, do đó các phép lai ghép đa nhiệm ít tỏ ra ưu thế của mình và chất lượng lời giải của MFEA.

Khi so sánh hiệu quả của mạng 2 lớp rộng với mạng nhiều lớp hơn nhưng sâu (3, 4, 5 lớp) như Hình V.1 với Hình V.7 hay trong Hình V.3 với Hình V.9, ..., ta thấy rằng các mạng sâu và hẹp có hiệu quả tốt tương đương với mạng nông và rộng hơn dù cho các mạng sâu hẹp ít tham số hơn nhiều. Điều này có được là do việc tăng độ sâu của mạng giúp tăng độ phức tạp của hàm xấp xỉ biểu diễn nên khớp dữ liệu học tốt hơn.

Tổng kết thực nghiệm, ta thấy rõ hiệu quả của chất lượng mạng ANN học bởi các phương pháp tiếp cận dựa trên MFEA đặc biệt là MFEA-II. Tuy nhiên, để có được chất lượng lời giải tất MFEA-II phải trả giá về mặt thời gian. Trung bình thời gian thực nghiệm trên một bộ dữ liệu của MFEA-II sẽ nhiều gấp 3 lần so với CEA và 2.5 lần nếu so với MFEA-I.

\subsection{Bài toán huấn luyện nhiều mô hình học tăng cường}

Với bộ dữ liệu thực nghiệm đơn giản Acrobot, các thuật toán đều có thể giải ngay ở những thế hệ đầu tiên, và mức độ chênh lệch giữa tốc độ hội tụ và mức độ tập trung kết quả cuối cùng cũng không thể hiện rõ sự chênh lệch giữa từng giải thuật.
Với bộ dữ liệu thực nghiệm phức tạp hơn như PixelCopter, FlappyBird, kết quả thực nghiệm trên các bộ dữ liệu học tăng cường chỉ rõ ưu điểm của cách tiếp cận của tiến hóa đa nhiệm. Về cả 3 mục tiêu chí so sánh giữa kết quả thực nghiệm, tốc độ hội tụ, mức độ phân bố kết quả cuối cùng thì kết quả được đưa ra bởi các thuật toán tiến hóa đa nhiệm đều vượt trội như trong hình \ref{fig:FLP_conv}, \ref{fig:FLP}, \ref{fig:PixelCopter_conv} và \ref{fig:PixelCopter}. Điều này có thể lý giải bởi với các bài toán có cấu trúc gần tương đồng với nhau việc trao đổi tri thức giữa các tác vụ giúp tăng tốc độ tối ưu trên từng tác vụ.
Và khi so sánh giữa các giải thuật tiến hóa đa nhiệm với nhau thì MFEA-II thể hiện sự vượt trội so với MFEA-I nhờ ưu thế có thể xác định được mức độ chia sẻ tri thức phù hợp tại từng thời điểm.