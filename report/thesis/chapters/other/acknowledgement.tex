Đồ án này sẽ không thể hoàn thiện nếu không có lời khuyên và sự hỗ trợ nhiệt tình mà tôi được nhận từ giảng viên hướng dẫn của tôi - PGS.TS Huỳnh Thị Thanh Bình. Mặc dù giáo sư bận rộn trong việc hỗ trợ nhiều nghiên cứu sinh, sinh viên, nhưng cô vẫn dành thời gian quý báu để giúp đỡ tôi nghiên cứu tốt hơn. Cùng với đó cô cũng đã tạo ra một môi trường nghiên cứu tuyệt vời để tôi và các sinh viên khác được cạnh tranh, cải thiện kết quả trong học tập và nghiên cứu. Tôi muốn bày tỏ lòng biết ơn của mình đến giáo sư Bình và chúc cô ngày càng thành công hơn trong công việc nghiên cứu, hỗ trợ sinh viên, cũng như thành công trong cuộc sống. Bên cạnh đó tôi cũng xin bày tỏ lòng biết ơn sâu sắc tới T.S Đinh Viết Sang đã nhiệt tình hỗ trợ và đưa ra những định hướng, lời khuyên trong suốt quá trình tôi thực hiện đồ án này.

Qua 5 năm học tập và nghiên cứu tại đại học Bách Khoa Hà Nội, tôi muốn cảm ơn những người thầy cô với kiến thức và lòng nhiệt huyết đã tham gia giảng dạy, giúp định hình tri thức và niềm tin trong tôi như hiện tại.

Đồ án này là tổng hợp của rất nhiều thời gian làm việc với mọi người trong phòng thí nghiệm Mô hình hóa, mô phỏng và tối ưu. Ở đây tôi được làm việc với nhiều sinh viên và các anh chị nghiên cứu sinh khác, tôi muốn cảm ơn họ những người đã khuyến khích tôi kiên trì với con đường hiện tại và trong tương lai đầy thử thách. Đặc biệt tôi muốn cảm ơn anh Thành lớp ICT K59, người luôn hỗ trợ và động viên giúp tôi vượt qua những khó khăn khi bước vào công việc nghiên cứu.

Cuối cùng, tôi muốn cảm ơn những người bạn đặc biệt tại lớp Việt Nhật K60C đã luôn ở đồng hành với tôi trong suốt 5 năm học tập dưới mái trường Bách Khoa Hà Nội. 

\begin{flushright}
\begin{minipage}[t]{0.5\textwidth}
\begin{center}
  \textit{Hà Nội}, ngày 20 tháng 06 năm 2020\\
  
  \textit{Hoàng Minh Quang}
\end{center}
\end{minipage}
\end{flushright}

\pagebreak