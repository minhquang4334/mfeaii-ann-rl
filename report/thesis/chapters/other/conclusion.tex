\textbf{Về mặt lý thuyết, đồ án đã trình bày được các nội dung sau}:
\begin{itemize}
    \item Tổng quan về mạng neural: mô hình, ý nghĩa toán học, các thuật toán lan truyền tham số cơ bản trong ANN.
    \item Tổng quan về tối ưu hóa liên tục, định nghĩa bài toán cùng một số phương pháp giải, tối ưu đa mục tiêu và tối ưu đa nhiệm.
    \item  Tổng quan về giải thuật tiến hóa đa nhiệm: cơ sở giải thuật tiến hóa cho bài toán tối ưu liên tục, cơ sở của giải thuật tiến hóa đa nhiệm.
    \item Đưa ra mối liên hệ giữa huấn luyện ANN và bài toán tối ưu. Đưa ra động lực cho giải thuật tiến hóa đa nhiệm áp dụng vào huấn luyện ANN.
    \item Đề xuất giải thuật học cho ANN dựa trên giải thuật tiến hóa đa nhiệm.
    \item Đưa ra mối liên hệ giữa ANN và áp dụng vào bài toán huấn luyện mô hình học tăng cường. Phương pháp áp dụng ý tưởng tiến hóa huấn luyện mô hình học tăng cường.
    \item Đề xuất giải thuật học cho ANN áp dụng vào bài toán huấn luyện mô hình học tăng cường dựa trên giải thuật tiến hóa đa nhiệm.
\end{itemize}
\textbf{Về mặt thực nghiệm, luận văn đã thu được một số kết quả}:
\begin{itemize}
    \item Cài đặt giải thuật đề xuất và thực nghiệm với dữ liệu phân loại nhị phân.
    \item Cài đặt giải thuật đề xuất và thực nghiệm với bài toán huấn luyện mô hình học tăng cường.
    \item So sánh kết quả thực nghiệm giải thuật đề xuất với cách tiếp cận tiến hóa đơn nhiệm và so sánh mô hình đa nhiệm ước lượng hệ số trao đổi trực tuyến với tiến hóa đa nhiệm thông thường.
    \item Tổng kết thực nghiệm đưa ra đánh giá, nhận xét và kết luận tính ưu nhược của giải thuật đề xuất.
\end{itemize}
\textbf{Các vấn đề còn tồn đọng chưa giải quyết}:
\begin{itemize}
    \item Dữ liệu thực nghiệm so sánh chưa đủ đa dạng.
    \item Chưa đánh giá đầy đủ hiệu quả của phương pháp tiếp cận dựa trên phương pháp đa nhiệm ước lượng hệ số trao đổi trực tuyến FEA-II và phương pháp tiến hóa đa nhiệm thông thường MFEA-I.
    \item Từ những tồn đọng đó, tác giả đề xuất mở rộng việc áp dụng nền tảng của MFEA-II cho các mô hình ANN khác nhau và trên nhiều bộ dữ liệu khác nhau.
\end{itemize}

\textbf{Các đề xuất hướng nghiên cứu trong tương lai mà tác giả có thể hướng đến}:
\begin{itemize}
    \item Mở rộng việc áp dụng MFEA-II cho các mô hình ANN khác nhau
và trên nhiều bộ dữ liệu khác nhau.
    \item Tiếp tục các nghiên cứu cơ bản đối với MFEA-II. Tập trung vào
giải quyết vấn đề khi nào nên dùng đa nhiệm với những độ đo thực tế hơn.
    \item Nghiên cứu phát triển các chiến lược tự thích nghi và điều chỉnh tham số.
    \item Nghiên cứu áp dụng MFEA-II vào giải quyết các bài toán học tăng cường phức tạp hơn.
\end{itemize}