
\subsubsection{Số chiều của không gian tìm kiếm $n$} Tham số này là số lượng tham số của bài toán. Ví dụ trong mmoojt bài toán tối ưu hóa liên tục có không gian tìm kiếm $x \in \mathbb{R}^n$. Sẽ có nhiều tham số khác phụ thuộc vào tham số $n$ này như xác suất cho đột biến xảy ra có thể tính dưới công thức $p_m=1/n$. Bên cạnh đó, số chiều của không gian tìm kiếm các lớn thì sẽ yêu cầu kích thước của quần thể $\mu$ lớn hơn, kích thước quần thể con cái $\lambda$ cũng lớn hơn.
    
\subsubsection{Kích thước quần thể $\mu$} Tham số này là số lượng cá thể trong quần thể. Nói chung thì không có quy tắc nào trong việc lựa chọn $\mu$ nhưng nếu $\mu$ quá nhỏ thì nhiều khả năng sẽ không thể giải quyết được bài toán. Nếu $\mu$ quá lớn thì thuật toán sẽ tốn rất nhiều chi phí tính toán.

\subsubsection{Kích thước quần thể con cái $\lambda$} Tham số này là số lượng cá thể mới được sinh ra trong mỗi thế hệ. 

\subsubsection{Tham số khác}
    \begin{itemize}
        \item Xác suất đột biến $p_m$
    \end{itemize}