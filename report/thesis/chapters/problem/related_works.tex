Sử dụng tiến hóa đa nhiệm để huấn luyện ANN là chủ đề nghiên cứu còn tương đối mới mẻ. Cho đến nay đã có nghiên cứu của Chandra et al \cite{chandra2018evolutionary} áp dụng tiến hóa đa nhiệm cho mô hình ANN với 1 lớp ẩn [53]. Trong bài báo này, Chandra et al sử dụng một phương pháp mã hóa đơn giản để mã hóa trực tiếp mô hình mạng.
Tuy nhiên, cách mã hóa này còn hạn tồn tại những chế nhất định cho dù kết quả thực nghiệm chỉ ra là rất đáng quan tâm. Bên cạnh đó có một số nghiên cứu khác như là áp dụng một giải thuật thay đổi của MFEA-I là MFEA-UDA với điều chỉnh giá trị của hệ số sinh sản ngẫu nhiên của Tuan NQ and Thanh Le\cite{nqtuan} áp dụng vào huấn luyện các mô hình ANN nhiều lớp. Các phương pháp này vẫn gặp phải một vấn đề đó là chưa giải quyết triệt để những hạn chế của thuật toán tiến hóa đa nhiệm khi mức độ trao đổi thông tin giữa các tác vụ vẫn dựa vào yếu tố ngẫu nhiên. 

Từ động lực này và phương pháp tiến hóa đa nhiệm với ước lượng hệ số trao đổi trực tuyến đã trình bày ở trên, đồ án này kế thừa cách mã hóa ANN đó để đề xuất phương pháp mã hóa số thực của Tuan NQ and Thanh Le cho ANN nhiều lớp một cách tổng quát, đồng thời phát triển một thuật toán dựa trên MFEA-II nhằm thực hiện huấn luyện các ANN khác cấu trúc đồng thời.